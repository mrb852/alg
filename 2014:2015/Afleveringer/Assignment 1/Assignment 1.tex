\documentclass[12pt]{article}
\usepackage{amsmath} % flere matematikkommandoer
\usepackage{amssymb}
\usepackage[utf8]{inputenc} % æøå
\usepackage[T1]{fontenc} % mere æøå
\usepackage[danish]{babel} % orddeling
\usepackage{verbatim} % så man kan skrive ren tekst
\usepackage[all]{xy} % den sidste (avancerede) formel i dokumentet
\usepackage{graphicx}

\renewcommand{\baselinestretch}{1.5} 

\title{Algorithms Assignment 1}
\author{Christian Enevoldsen\\MRB852\\}



\begin{document}
\maketitle

\newpage
\section{Master method}

The master method can be applied to functions with form

\[T(n) = aT(\frac{n}{b}) + f(n)\]

Where 
\[a \ge 1\]
\[b \ge 2 \]

The master method is applied to the following recurrences

\begin{enumerate}
\item $p(n) = 8p(n / 2) + n^2$
\item $p(n) = 8p(n / 4) + n^3$
\item $p(n) = 10p(n / 9) + n \log _2 n$
\end{enumerate}

For the 1st recurrence we have $\begin{cases}
	a: 8 \\
	b: 2 \\
\end{cases} $

Since $f(n) = n^2$ and $n^{\log _b a} = n^{\log _2 8} = n^3$ 
there exists some constant $\epsilon = 4$ 

such that   $ n^2 = O(n^{\log _b a - \epsilon}) \Rightarrow \underline{\underline{ p(n) \in \Theta(n^3)}}$\\\\

For the 2nd recurrence we have $\begin{cases}
	a: 8 \\
	b: 4 \\
	f(n): n^3
\end{cases} $

There exists a constant $\epsilon = 56$ such that $n^3 = \Omega (n^{\log _4 8 + \epsilon})$

Since $af(n/b) \leq cf(n) \Leftrightarrow 8(n/4)^3 \leq cn^3$ we get $\underline{\underline{p(n) \in \Theta(n^3)}}$


For the 3rd recurrence we have $\begin{cases}
	a: 10 \\
	b: 9 \\
	f(n): n\log _2 n 
\end{cases} $

There exists a constant $\epsilon > 1$ such that $n\log _2 n = \Omega (n^{\log _9 10 + \epsilon})$

Since $af(n/b) \leq cf(n) \Leftrightarrow 10\frac{n}{9}\log _2 \frac{n}{9} \leq cn\log _2 n$ we get $\underline{\underline{p(n) \in \Theta(n^{log_9 10})}}$

\section{Substitution method}

The substitution method is applied to the following recurrences

\begin{enumerate}
\item $p(n) = p(n/2) + p(n/3) + n$
\item $p(n) = \sqrt{n} \cdot p(\sqrt{n}) + \sqrt{n}$
\end{enumerate}

We make a guess that the first recurrence has running time $O(n\log _2 n)$
This can be done by either just guessing or with a recurrence tree. 
The guess is done with a recurrence tree however the illustration is discarded
in this report.



\end{document}
