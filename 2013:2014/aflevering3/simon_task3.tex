\documentclass[12pt]{article}
\usepackage{amsmath} % flere matematikkommandoer
\usepackage{amssymb}
\usepackage[utf8]{inputenc} % æøå
\usepackage[T1]{fontenc} % mere æøå
\usepackage[danish]{babel} % orddeling
\usepackage{listings}
\usepackage{color}
\usepackage{enumerate}
\usepackage{algpseudocode}
\usepackage{algorithm}
\usepackage{mathtools}
\definecolor{dkgreen}{rgb}{0,0.6,0}
\definecolor{gray}{rgb}{0.5,0.5,0.5}
\definecolor{mauve}{rgb}{0.58,0,0.82}
\DeclarePairedDelimiter\ceil{\lceil}{\rceil}
\DeclarePairedDelimiter\floor{\lfloor}{\rfloor}

\lstset{frame=tb,
 language=Java,
 aboveskip=3mm,
 belowskip=3mm,
 showstringspaces=false,
 columns=flexible,
 basicstyle={\small\ttfamily},
 numbers=none,
 numberstyle=\tiny\color{gray},
 keywordstyle=\color{blue},
 commentstyle=\color{dkgreen},
 stringstyle=\color{mauve},
 breaklines=true,
 breakatwhitespace=true
 tabsize=3
}

\newcommand{\abs}[1]{\lvert#1\rvert}
\newcommand{\norm}[1]{\lVert#1\rVert}

\newenvironment{amatrix}[1]{%
  \left(\begin{array}{@{}*{#1}{c}|c@{}}
}{%
  \end{array}\right)
}

{\setlength{\parindent}{0 cm}

\title{Assignment 3 - AlgDat}
\author{Simon Warg BCS315}

\begin{document}
\maketitle

\section*{Task 1}
\begin{algorithm}
 \caption{CanDistribute}

 \begin{algorithmic}
   \For  {$i=1$ to $n-1$}
        \State $distance = p_{i+1}-p_i$
         \If{$b_i < \hat{b}$}
                \State $b_{i+1} = b_{i+1}- (\hat{b}-b_i) - 2*distance$
                \State $b_i = \hat{b}$
        \EndIf
         \If{$b_i > \hat{b}$}
                \State $b_{i+1} = max(0, b_i - \hat{b} - 2*distance)$
                \State $b_i = \hat{b}$
        \EndIf
   \EndFor \\
\Return $b_n \geq \hat{b}$
\end{algorithmic}
\end{algorithm}

\section*{Task 2}
Hvis $b_1 < \hat{b}$, så vil den billigste, og dermed den beste, løsningen være at tage øl fra $b_2$. Hvis $b_2$ derefter får mindre end $\hat{b}$ øl så vil den endeste måde være at tage øl fra $b_3$. Det andet case; hvis $b_1 > \hat{b}$, så vil den beste løsningen være at sørge for at $b_2$ får så mange øl fra $b_1$ som mulig indtil $b_1 = \hat{b}$. Dermed gælder at for hver $b_i$ så er $b_1...b_{i-1} \leq \hat{b}$. Hvis den sidste bar $b_n \geq \hat{b}$ ved vi at alle barer kan have $\hat{b}$ øl.

\section*{Task 3}
\begin{algorithm}
 \caption{Find maximum number of beer that can be distributed among all bars}

 \begin{algorithmic}
        \State $left = 0$
        \State $right = B$
   \While{$right-left > 1$}
        \If{$CanDistribute(\floor*{\frac{right+left}{2}})$}
                \State $left = \floor*{\frac{right+left}{2}}$
        \Else
                \State $right = \floor*{\frac{right+left}{2}}$
        \EndIf
\EndWhile
\If{$CanDistribute(right)$}
        \State \Return right
\Else
        \State \Return left
\EndIf

\end{algorithmic}
\end{algorithm}

Funktionen kører i $O(\log{B}) + O(n)$ fordi i hvert iteration så tjekkes det hvis maximum $m$, $0 < m < B$, i et søgområde der halveres efter hver iteration. Hver gang den søger i søgområdet bruger den $O(n)$ tid, altså kører funktion i $O(n\log B)$.

Funktionen er korrekt siden den hver gang tjekker hvis maximum er under eller over $B/2$. Er den under så fortsætter den at undersøge om maximum er under eller over $B/2-B/4$, er den over så tjekker den hvis maximum er over eller under $B/2+B/4$ indtil søgområdet er kun to tal, hvor den tjekker begge for maximum.

\end{document}
