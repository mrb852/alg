\documentclass[12pt]{article}
\usepackage{amsmath} % flere matematikkommandoer
\usepackage{amssymb}
\usepackage[utf8]{inputenc} % æøå
\usepackage[T1]{fontenc} % mere æøå
\usepackage[danish]{babel} % orddeling
\usepackage{verbatim} % så man kan skrive ren tekst
\usepackage[all]{xy} % den sidste (avancerede) formel i dokumentet
\usepackage{graphicx}

\title{LinAlgDat projekt D}
\author{Nicklas Warming Jacobsen\\Studenr. QMR656\\Holdnr. 2}

\begin{document}
\maketitle

\newpage
\section*{Task 1}
\textbf{1: }For $p(n)=8p(n/2)+n^2$ gælder det at $p(n)=\Theta(n^3)$.
Dette kan man se ved at bruge Theorem 4.1 case 1, hvor det skal gælde at for $\epsilon > 0$:
\begin{flalign*}
  f(n)&=O(n^{log_ba-\epsilon})
\end{flalign*}
$\epsilon$ bliver valgt til $\epsilon=1$
\begin{flalign*}
  n^2&=O(n^{log_28-1}) \Leftrightarrow n^2=O(n^2)
  \Downarrow\\
  p(n) &=  \Theta(n^{log_28})=\Theta(n^3)
\end{flalign*}
\textbf{2: }For $p(n)=8p(n/4)+n^3$ gælder det at $p(n)=\Theta(n^3)$.
Dette kan man se ved at bruge Theorem 4.1 case 3, hvor det skal gælde at for $\epsilon > 0 \land c < 1$:
\begin{flalign*}
  f(n)&=\Omega(n^{log_ba+\epsilon}) \land a\cdot f(n/b) \leq c\cdot f(n)
\end{flalign*}
$\epsilon$ bliver valgt til $\epsilon=1,5$
\begin{flalign*}
  n^3&=\Omega(n^{log_48-15})\\
  &\Updownarrow\\
  n^3&=\Omega(n^3)
\end{flalign*}
Vi vælger $c=1/8$ og får:
\begin{flalign*}
  8\left(\frac{n}{4}\right)^3 &\leq \frac{1}{8}\cdot n^3\\
  &\Updownarrow\\
  8 \frac{n^3}{64} &\leq \frac{1}{8}\cdot n^3\\
  &\Updownarrow\\
  \frac{n^3}{8} &\leq \frac{n^3}{8}
\end{flalign*}
\textbf{3: }For  $p(n)=10p(n/9)+n\cdot log_2n$ gælder det at $p(n)=\Theta(n^{log_910}log_2n)\approx\Theta(n^{1,048}log_2n)$.
Dette kan man se ved at bruge Theorem 4.1 case 3, hvor det skal gælde at:
\begin{flalign*}
  f(n)=\Theta(n^{log_ba})
\end{flalign*}
\begin{flalign*}
  n\cdot log_2n&=\Theta(n^{log_910})\\
  n\cdot log_2n&=\Theta(n^{1,048})
\end{flalign*}
\section*{Exam subject outline - Nicklas Jacobsen qmr656}
\textbf{Divide}: Split problemet i mindre problmer\\
\textbf{Conquer}: Når problemerne er små nok, så løs dem på en triviel måde\\
\textbf{Combine}: Kombinerer løsningerne til en stor løsning på det samlede problem
\subsection*{Substitutions metode}
Metoden består af 2 step:\\
1: Gæt en løsning
2: Matematisk induktions til bevis at løsningen er rigtig.
$$
T(n)=2T(n/2)+n
$$
Vi gætter $T(n)=O(n\cdot lg(n))$
Vi substituerer ind:
\begin{flalign*}
  T(n) &\leq 2(c(n/2)lg(n/2))+n\\
  &\leq cn\cdot lg(n/2)+n\\
  &=cn\cdot lg(n)-cn\cdot lg(2)+n\\
  &=cn\cdot lg(n)-cn+n\\
  &\leq cn\cdot lg(n)
\end{flalign*}
Bemærk det kun gælder for $n>1$
\subsection*{Master method}
Hvis vi har en recurrence på følgende form:
\begin{flalign*}
  T(n)&=aT(n/b)+f(n)
\end{flalign*}
Så har har $T(n)$ følgende asymptotiske grænser.\\\\
1: Hvis $f(n)=O(n^{log_ba-\epsilon})$ ved en konstant $\epsilon>0$, så er det ensbetydende med at $T(n)=\Theta(n^{log_ba})$\\\\
2: Hvis $f(n)=\Theta(n^{log_ba})$, så $T(n)=\Theta(n^{log_ba}lg(n))$\\\\
3: hvis $f(n)=\Omega(n^{log_ba+\epsilon})$ for en konstant $\epsilon>0$, og hvis $a\cdot f(n/b) \leq c\cdot f(n)$ for en konstant $c<1$, så $T(n)=\Omega(f(n))$
\subsection*{Eksempel - Merge-sort}
Merge-sort er en Divide-and-Conquer sorterings algoritme og foregår i $O(n\cdot lg(n)$ tid.
Algoritmen deler listen af tal op rekursivt til mindre del-lister indtil del-listerne har ét element, og derved bliver anset som været sorteret.
For hver del-liste samler (merger) algoritmen listerne til en større del-liste, ved linært at samligne elementerne i listerne. Når der kun er en del-liste tilbage, er listen sorteret.

Merge-sort kan deles op i to del-algoritmer: 1. Merge som er $O(n)$ og 2. Divide som er $O(lg(n)$. Og den rekursive form er derved $T(n)=2(n/2)+n$.
\end{document}
