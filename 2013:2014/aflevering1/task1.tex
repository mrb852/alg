\documentclass[12pt]{article}
\usepackage{amsmath} % flere matematikkommandoer
\usepackage{amssymb}
\usepackage[utf8]{inputenc} % æøå
\usepackage[T1]{fontenc} % mere æøå
\usepackage[danish]{babel} % orddeling
\usepackage{verbatim} % så man kan skrive ren tekst
\usepackage[all]{xy} % den sidste (avancerede) formel i dokumentet
\usepackage{graphicx}

\begin{document}

\section*{Task 1}
\textbf{1: }For $p(n)=8p(n/2)+n^2$ gælder det at $p(n)=\Theta(n^3)$.
Dette kan man se ved at bruge Theorem 4.1 case 1, hvor det skal gælde at for $\epsilon > 0$:
\begin{flalign*}
  f(n)&=O(n^{log_ba-\epsilon})
\end{flalign*}
$\epsilon$ bliver valgt til $\epsilon=1$
\begin{flalign*}
  n^2&=O(n^{log_28-1}) \Leftrightarrow n^2=O(n^2)
  \Downarrow\\
  p(n) &=  \Theta(n^{log_28})=\Theta(n^3)
\end{flalign*}
\textbf{2: }For $p(n)=8p(n/4)+n^3$ gælder det at $p(n)=\Theta(n^3)$.
Dette kan man se ved at bruge Theorem 4.1 case 3, hvor det skal gælde at for $\epsilon > 0 \land c < 1$:
\begin{flalign*}
  f(n)&=\Omega(n^{log_ba+\epsilon}) \land a\cdot f(n/b) \leq c\cdot f(n)
\end{flalign*}
$\epsilon$ bliver valgt til $\epsilon=1,5$
\begin{flalign*}
  n^3&=\Omega(n^{log_48-15})\\
  &\Updownarrow\\
  n^3&=\Omega(n^3)
\end{flalign*}
Vi vælger $c=1/8$ og får:
\begin{flalign*}
  8\left(\frac{n}{4}\right)^3 &\leq \frac{1}{8}\cdot n^3\\
  &\Updownarrow\\
  8 \frac{n^3}{64} &\leq \frac{1}{8}\cdot n^3\\
  &\Updownarrow\\
  \frac{n^3}{8} &\leq \frac{n^3}{8}
\end{flalign*}
\textbf{3: }For  $p(n)=10p(n/9)+n\cdot log_2n$ gælder det at $p(n)=\Theta(n^{log_910}log_2n)\approx\Theta(n^{1,048}log_2n)$.
Dette kan man se ved at bruge Theorem 4.1 case 3, hvor det skal gælde at:
\begin{flalign*}
  f(n)=\Theta(n^{log_ba})
\end{flalign*}
\begin{flalign*}
  n\cdot log_2n&=\Theta(n^{log_910})\\
  n\cdot log_2n&=\Theta(n^{1,048})
\end{flalign*}

\end{document}
