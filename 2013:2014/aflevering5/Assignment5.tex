\documentclass[12pt]{article}
\usepackage{amsmath} % flere matematikkommandoer
\usepackage{amssymb}
\usepackage[utf8]{inputenc} % æøå
\usepackage[T1]{fontenc} % mere æøå
\usepackage[danish]{babel} % orddeling
\usepackage{listings}
\usepackage{color}
\usepackage{enumerate}
\usepackage{algpseudocode}
\usepackage{algorithm}
\usepackage{mathtools}
\definecolor{dkgreen}{rgb}{0,0.6,0}
\definecolor{gray}{rgb}{0.5,0.5,0.5}
\definecolor{mauve}{rgb}{0.58,0,0.82}
\DeclarePairedDelimiter\ceil{\lceil}{\rceil}
\DeclarePairedDelimiter\floor{\lfloor}{\rfloor}
 
\lstset{frame=tb,
 language=Java,
 aboveskip=3mm,
 belowskip=3mm,
 showstringspaces=false,
 columns=flexible,
 basicstyle={\small\ttfamily},
 numbers=none,
 numberstyle=\tiny\color{gray},
 keywordstyle=\color{blue},
 commentstyle=\color{dkgreen},
 stringstyle=\color{mauve},
 breaklines=true,
 breakatwhitespace=true
 tabsize=3
}
 
\newcommand{\abs}[1]{\lvert#1\rvert}
\newcommand{\norm}[1]{\lVert#1\rVert}
 
\newenvironment{amatrix}[1]{%
  \left(\begin{array}{@{}*{#1}{c}|c@{}}
}{%
  \end{array}\right)
}
 
{\setlength{\parindent}{0 cm}
 
\title{Assignment 5 - AlgDat}
\author{Simon Warg\\Nicklas Jacobsen\\Robert Rasmussen\\Christian Enevoldsen}
 
\begin{document}
\maketitle 

\newpage
\section*{Task 1}
Vi beviser at et MST $T$ er en del af den oprinelige graf G ved hjælp af et modstridsbevis. Så vil vise at $T$ ikke nødvendigvis er en del af $G$. Antag at $T$ ikke er en del af $G$, således at der findes en kant $e$ i $T$ der ikke findes i mængden af kanterne $E$ i $G$. Men da $T$ er et MST så er det per definition en delmængde $T \subseteq E$. Vi ankommer således til et modstrid, og vores antagelse at $T \nsubseteq E$ må være falsk. Vi kan derfor konkludere at $T$ er en del af $G$.
 
\section*{Task 2}
Let $G$ be the original graph\\
Let $H$ be a matrix of shortest path distance between breweries\\
Let $MST$ be the graph created by Prim's Algorithm\\
getEdge: retrieves a random edge of weight findSize from matrix m,
  which is not in MST
\begin{algorithm}
 \caption{FindLastEdge}
 \begin{algorithmic}
	\Function{findLastEdge}{G, m, MST}
		\State findSize = $G.weight - MST.weight$
		\State lastEdge = getEdge(findSize, m, MST)
		\State G2.add(lastEdge)
	\EndFunction
\end{algorithmic}
\end{algorithm}
\newpage
\section*{Exame outline - MST - Nicklas W. Jacobsen}
\subsection*{Definition af Minimum spanning tree}
Et MST af en givet graf $G$, er en non-cyklisk del-graf $T$ som forbinder alle knuder i $G$, med andre ord er $T$ et træ som udspænder $G$.
\subsection*{Dyrkning af en MST for en givet graf $G$}
Hvis vi vil finde en MST af en givet graf $G=(V,E)$, med en vægt givet som $w : E \rightarrow \mathbb{R}$, definerer vi en $A$ som værende et sub-sæt af en eller anden MST for $G$.
Herefter følger vi en grådig metode, hvor vi tilføjer en sikker kant til $A$,  en efter en indtil vi har en MST af $G$. Vi definerer en sikker kant for $A$, som værende en kant der ikke bryder invarianten.
\begin{algorithm}
 \caption{GENERIC-MST}
 \begin{algorithmic}
	\Function{GenericMST}{G, w}
	\State $A = \emptyset$
	\While{A does not form a MST, find an edge (u, v) this is safe for A}
		\State $A = A \cup {(u,v)}$
	\EndWhile
	\State Return $A$
	\EndFunction
\end{algorithmic}
\end{algorithm}
\paragraph{Initialisering: } Efter linje 2, holder $A$ trivielt invarianten.
\paragraph{Vedligeholdelse: } I hver iteration på linje 3 og 4, bliver invarianten holdt da det kun er sikre kanter der bliver tilføjet til $A$.
\paragraph{Terminering: } Interationen stopper når $A$ er en MST af $G$, og $A$ bliver retuneret. Derfor må funktionen retunerer MST af $G$.
\subsection*{Kruskal's algoritme}
\begin{algorithm}
 \caption{MST-KRUSTAL}
 \begin{algorithmic}
	\Function{MST-KRUSTAL}{G, w}
	\State $A = \emptyset$
	\For{each vertex $e \in G.V$}
		Make-Set(v)
	\EndFor
	\State Sort G.E in ascending order by weight $w$
	\For{each $(u,v) \in G.E$}
		\If{Fïnd-Set(u) $\neq$ Find-Set(v)}
			\State $A = A \cup {(u,v)}$
			\State Union(u,v)
		\EndIf
	\EndFor
	\State Return $A$
	\EndFunction
\end{algorithmic}
\end{algorithm}
Algoritmen bruger ``dis-join`` datatype til at holde styr på hvorvidt om det sikkert at tilføje en kant til $A$. Hvis det ikke er forsætter den til næste kant
\section*{Exam outline - Robert Rasmussen}
\subsection*{points}
\begin{itemize}
\item What is Minimum Spanning Trees
\item Prim's algorithm
\item Kruskal's algorithm
\end{itemize}

\subsection*{Problem Instance}
A weighted graph to run Prim's on.
\section*{Exam outline - Christian Enevoldsen}
\subsection *{Disposition}
\begin{itemize}
\item Definition
\item Generisk Algoritme
\item Prim's algoritme: Tidskompleksitet
\item Kruskal's algoritme:  Tidskompleksitet
\item Problem instans: Kruskal's algoritme paa en vaegtet graf.
\end{itemize}
\section*{Exam-outline - Simon Warg}
\subsection*{Minimum spanning trees}
A MST is a weighted tree $T$ containing the nodes $T.V$ of a graph $G = (V, E)$ and a subset of the graph's edges $T.E \subseteq G.E$ such that the sum of G's weight $\sum\limits_{(u,v) \in G.E} \omega{(u,v)}$ is minimized.

\subsubsection*{Prim's algorithm}
A MST can be optained by following Prim's algorithm. It maintains a priority queue $Q$  where all not-yet visited verticies resides. It then iterates Q until it's empty. Each iteration, it picks the edge with lowest edge connecting cut $T$ and $V - T$ and then add it to $T$.

\end{document}
